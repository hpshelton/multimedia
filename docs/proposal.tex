\documentclass[12pt]{article}

\usepackage{hyperref}

\hypersetup{
	colorlinks = true,
	citecolor = black,
	linkcolor = black,
	urlcolor = blue,
	filecolor = blue,
}

\setlength{\topmargin}{-1in}
\setlength{\oddsidemargin}{0in}
\setlength{\evensidemargin}{0in}
\setlength{\textwidth}{6.5in}
\setlength{\textheight}{8.5in}

\newcommand{\degrees}{\ensuremath{^\circ}}
\renewcommand{\vec}[1]{\ensuremath{\mathbf{#1}}}

\newcommand{\hpsheader}[1]{
   \noindent      
   \hbox to \textwidth {\hfill H. Parker Shelton}
   \vspace{0mm}
   \hbox to \textwidth {\hfill Adam Feinstein}
   \vspace{0mm}
   \hbox to \textwidth {\hfill \today}
   \vspace{0mm}
   \begin{center}
       \hbox to \textwidth {{\Large \hfill #1 \hfill} }
   \end{center}
}

\begin{document}
\hpsheader{Final Project Proposal}

\section{Abstract}
We will implement a basic cross-platform parallelized video and image editor in C++ that will enable rotation, cropping, scaling, color-correction, and assorted other features such as edge detection, motion detection, blurring, and wavelet quantization compression.

\section{Description}
The project will be distributed as a stand-alone, cross-platform application built using \href{http://qt.nokia.com/}{Qt}, notably the GUI framework. Editing functions will be implemented in C++, supported by parallelization with NVIDIA's \href{http://www.nvidia.com/object/cuda_home_new.html}{CUDA}. Decoding will support qcif and pgm files and encoding will output ppc (parallel picture codec) and pvc (parallel video codec)  files encoded with Huffman, run-length, or arithmetic compression as well as motion estimation and frame difference. Both the encoder and decoder will be parallelized.

The application interface will contain a simple toolbar for performing each manipulation, with the possibility of a dialog box in which to specify different parameters. The main window will show the original image and the manipulated image side-by-side.

\section{Timeline}

\begin{description}
\setlength{\labelsep}{25pt}

\item[Mar 24] Project Proposal
\item[Mar 31] Application GUI
\item[Apr 07] Encoding/Decoding
\item[Apr 21] Parallelized Image Editing Features
\item[Apr 28] Parallelized Video Editing Features
\item[May 05] Project Presentation
\item[May 12] Final Project Report

\end{description}
\end{document}
